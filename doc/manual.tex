% London Law documentation
\documentclass[11pt,notitlepage]{article}
\usepackage{times}
\usepackage{fullpage}

% End preamble
%%%%%%%%%%%%%%%%%%%%%%%%%%%%%%%%%%%%%%%%%%%%%%%%%%%%%%%%%%%%%%%%%%%%%%%%%%%%%%%
\begin{document}
\title{London Law v0.2 User Manual}
\author{Paul J. Pelzl}
\date{February 15, 2005}
\maketitle

\section*{Introduction}
London Law is a networked multiplayer adaptation of the classic Scotland Yard
board game, published by Ravensburger in 1983.  One player takes on the
role of Mr. X, who must evade Scotland Yard by carefully concealing his
path through London.  Another one to five players control five detectives,
who must work together to locate Mr. X using the limited clues he leaves
behind.  This game has the unusual quality of asymmetry--the two ``teams''
have different goals and different abilities.

London Law is written in Python, and makes use of the wxPython GUI library.
As a result, it should be quite cross-platform.  It will certainly run on
GNU/Linux (my development platform), and ought to run on *BSD, Mac OS X,
Windows, and others.

\section*{Installation}
Before installing London Law, you should have installed 
%BEGIN LATEX
Python\footnote{http://www.python.org}
%END LATEX
%HEVEA \begin{rawhtml} <a href="http://www.python.org">Python</a> \end{rawhtml} 
2.3 or greater and 
%BEGIN LATEX
wxPython\footnote{http://www.wxpython.org}
%END LATEX
%HEVEA \begin{rawhtml} <a href="http://www.wxpython.org">wxPython</a> \end{rawhtml} 
2.4.x or 2.5.x.  (Mac users are advised to use wxPython 2.5.x, as its OS X support
is more mature.)  In addition, you will need the
%BEGIN LATEX
Twisted\footnote{http://www.twistedmatrix.com}
%END LATEX
%HEVEA \begin{rawhtml} <a href="http://www.twistedmatrix.com">Twisted</a> \end{rawhtml} 
networking library.  Most popular GNU/Linux and BSD distributions provide packages
for all of these components.

If you are using Windows, simply extract the London Law archive to a useful
location and disregard the remainder of this section.  If you are using a
Unix-like operating system, read on for installation instructions.  Note that
installation is optional; London Law may be executed straight out of the source tree from
the {\tt londonlaw} subdirectory.

I will assume you have received this program in the form of a source tarball, 
e.g. ``londonlaw-x.x.x.tar.gz''.  You have undoubtedly extracted this archive 
already (e.g. using ``{\tt tar xvzf londonlaw-x.x.x.tar.gz}'').  Enter the root of 
the londonlaw installation directory, e.g. ``{\tt cd londonlaw-x.x.x}'';  there should 
be a file called ``setup.py'' in this directory.  Become root before 
trying to perform the installation.  There are a couple of ways to install London Law:
\begin{enumerate}
\item {\bf Direct installation from source.}  You can copy all the necessary 
files to logical locations with the command ``{\tt python setup.py install}''.  

If you wish to choose a different installation prefix, you can use ``{\tt 
python setup.py install --prefix=PREFIX}'', where {\tt PREFIX} is your desired
installation root, for example ``/usr/local''.  (If you choose this option, 
your Python search path must include the corresponding directories.)
\item {\bf Installation via rpm.}  If you use an rpm-based distribution you 
can build an rpm package for London Law, which gives you the ability to remove the 
program easily (not that you would ever want to do that).  Create the rpm with 
the command ``{\tt python setup.py bdist\_rpm}''.  The package should be created 
within the ``dist'' subdirectory, and you can install it using \\
``{\tt rpm -Uvh londonlaw-x.x.x-x.noarch.rpm}''.

If you have a Debian-based system, then you could use {\tt alien} to create a .deb 
from this rpm. 
\end{enumerate}


\section*{Playing the game}
\subsection*{Setting Up}
London Law requires a game server, so start by executing ``{\tt london-server}'' on
a networked host; this command accepts an optional ``{\tt -pPORT}'' flag, which
causes the server to listen for connections on the specified port.  Now you and your 
gaming buddies can start your graphical clients by executing ``{\tt london-client}''.

Note: if you are using Windows, use the {\tt london-server.py} and  {\tt
london-client.py} files, which should be associated with Python.  These are
located in the {\tt londonlaw} subdirectory.  If you are using OS X, you must
launch the client with the special command ``{\tt pythonw london-client}'' in order
to permit the client to create graphics.

Upon launching the client, you will be presented with a connection window.  Enter
the hostname or IP address of the machine hosting the game server, provide a username
and a password of your choice (do not forget the password!), and click ``Connect''.

Next you will be presented with a list of games.  You can select a game and
join it, or click ``New Game'' to create your own.  Once inside the game room,
click ``Choose Team'' if you want to select whether you will play as Mr. X or as the
detectives.  When you are satisfied with your choice, click ``Vote to Start''.
When (1) there are players on both teams and (2) all players have voted to
start, the game will begin.

If you disconnect from the server during gameplay, whether intentionally or not, you
may reconnect and rejoin the game at any time (provided you are using the same
username and password).  The server may also be shut down and restarted without loss of
game data.

\subsection*{Gameplay}
\subsubsection*{Overview}
There are 199 locations marked on the London map.  These locations are connected by a
transportation network composed primarily of taxi, bus and underground lines (colored 
yellow, green, and maroon, respectively).  There are also a few river routes,
rendered as dashed black lines.  All detectives begin with 10 taxi tickets, 8
bus tickets, and 4 underground tickets.  Mr. X has a transportation advantage:
in addition to having unlimited taxi, bus, and underground tickets, he also has
2 pink ``double move'' tickets and 5 black tickets.  The black tickets can be
used for any form of transportation, including the river routes; they are
useful for disguising Mr. X's movements.  The current ticket counts are
displayed at the side of each player icon.

Mr. X and the detectives take turns moving throughout the map, spending tickets
as they go.  Mr. X's location remains hidden for most of the game, and is
revealed only at turns 3, 8, 13, 18 and 24 (labeled with question marks in the
history window).  The detectives are allowed to see what type of ticket Mr. X
uses at every turn.  Mr. X's tickets and surfacing locations are displayed in
the history window at the left of the client screen.  Mr. X's goal is to evade
the detectives until the end of the game; the detectives, of course, are trying
to track him down.

The game ends when one of the following occurs:
\begin{itemize}
   \item Mr. X evades the detectives for 24 turns.  (Mr. X wins.)
   \item One of the detectives lands on Mr. X's location, or vice versa.  (Detectives win.)
   \item All the detectives are stuck--they have run out of useful tickets, or are
      unable to move without landing on another detective.  (Mr. X wins.)
\end{itemize}

\subsubsection*{Controls}
\begin{itemize}
   \item You can make a move either by clicking the ``Move'' button, or by
      double-clicking on a map location.  If you double-click on your desired
      destination, this information will be automatically entered into the Move
      dialog.
   \item You can drag the map around using the middle or right mouse buttons.
   \item Double-clicking on a player's icon will center the map on that player's
      location.
   \item The View menu may be helpful for getting a wider view of the map; there are
      options (and hotkeys) for viewing the map in fullscreen, zooming in and
      out, and hiding the history window.
\end{itemize}


\section*{Licensing}
London Law has been made available under the GNU General Public License (GPL), 
version 2.  You should have received a copy of the GPL along with this 
program, in the file ``COPYING''.


\section*{Credits}
I would like to express my thanks to:
\begin{itemize}
   \item Conor Davis, for significant contributions to the recent server rewrite
   \item Robin Dunn, for his excellent wxPython bindings (which really make wxWidgets
      programming a piece of cake)
   \item The Twisted Matrix hackers, for a very well designed (if somewhat massive)
      networking library
   \item David Schaller, for reintroducing me to Scotland Yard
\end{itemize}


\section*{Contact info}
London Law author: Paul Pelzl {\tt <pelzlpj@eecs.umich.edu>} \\
London Law website: {\tt http://www.eecs.umich.edu/\~{}pelzlpj/londonlaw} \\


\noindent
Feel free to contact me if you have bugs, feature requests\footnote{First have a look at the
included ``TODO'' list; I've got quite a few things in mind for future releases.}, patches, AI
opponents, etc.  I would also welcome volunteers
interested in packaging London Law for various platforms.

London Law is developed with the aid of the excellent 
%BEGIN LATEX
Arch RCS\footnote{http://www.gnu.org/software/gnu-arch/}.  
%@% END LATEX
%HEVEA \begin{rawhtml} <a href="http://www.gnu.org/software/gnu-arch/">Arch RCS</a>. \end{rawhtml} 
Interested developers are advised to track London Law development via my public repository: \\
\hspace*{2cm}{\tt pelzlpj@eecs.umich.edu--2005 $\backslash$ \\
\hspace*{3cm} http://www-personal.engin.umich.edu/\~{}pelzlpj/tla/2005} \\
London Law code may be found in branch {\tt londonlaw--main} .

Do you feel compelled to compensate me for writing London Law?  As a {\em poor, 
starving} graduate student, I will gratefully accept donations.  Please see \\
{\tt http://www.eecs.umich.edu/\~{}pelzlpj/londonlaw/donate.html} for more information.
\end{document}




% arch-tag: DO_NOT_CHANGE_cfd1f7bd-124e-447a-a21f-89ca519c7235 
